\documentclass{article}
\usepackage[utf8]{inputenc}
\title{Cognitive Science 1 Lecture Notes}
\author{Benjamin Lee}
\date{March 2018}

\begin{document}

\maketitle

\section{Lecture: Explaining Consciousness: UC Berkeley Philosophy Professor Geoffrey Lee }

Philosophy of Mind \\
 Empirical Psychology: the scientific study of the mind/brain \\
 Folk Psychology: our everyday ways of talking and thinking about the mind \\
 
How are the two ready? (main focus of Lee's work is philosophical thinking of the mind)
    Study of cog sci and neuroscience vs. (not needing a college level class to be adept about thinking and talking about the mind) audibly innate capacity to represent and think about other peoples minds without being taught about it. (ie. what other people believe, want to do, emotions, etc.) These ways are so familiar to us that we don't notice the peculiarity of them. 
    Ex: I know everyone came to lecture today because of their belief or want to learn or pass the class, not because I knew the physical composition of their body. 
    
Look at those two descriptions and how they can fit together in the same picture \\

\subsection{Consciousness}

\textbf{Nagel's Definiton:} A mental state is conscious just if there is "something it's like" to have that mental state \\
Other kinds of consciousness: self consciousness, access consciousness. \\ 
can only really point to the phenomenon, no real definition yet. \\ 
 
How can we tell whether other systems (Ai for example) are conscious? \\ 
Once they attain this high level consciousness as humans, how do we treat them? must have some moral decision in how to treat thme \\

 Can there be a science of consciousness?
 Reductionism consciousness is a special kind of complex processing in the brain. 
 
 Supervenience: no differences in consciousness without differences in the brain (consciousness if fully determined by the brain). 
 
 Neural Correlates of Consciousness: what are the neural bases in humans of particular kinds of conscious experience?
 
 Thalamus important for consciousness. USe of general anesthetics and how they work to discover conscious centers of the brain. 
 
 Ex: Binocular rivalry (tong et al Neuron 1990) 
    must be viewed through red/green glasses
    shifting percept 
    constant stimulus
Fusiform Face Area (FFA) 
Parahippocampal place area (PPA)


Two obstacles to explaining consciousness 
The Hard Problem 
The methodological Problem

\subsection{Learning is Moving in New Ways: \\ Building Educational Theory Through Design-Based Research on Embodied Mathematics Cognition and Instruction \\  \\ Embodied Design Research Laboratory: Dor Abrahamson}
intro \\
Learning Education: combined with cognitive science and teaching. \\
Designer: educational designer: lessons, gadgets, etc. 
Design based research: design for the future?
Mathematics Cognition and Instruction: how do people make sense of mathematics?
Building Educational Theory: theory, design, make, redevelop theory through mistakes

edrl.berkeley.edu \\

\textbf{Investigative Approach: Design-based Research: }

Conjecture-driven, iterative, empirical studies, wherein, theory and design, co-inscribe. \\
Creating designs for young people, young minds, for people to learn. \\
1837: Beginning of Kindergarden 
Friedrich Frobel, Maria Montessori, Hans Freudentahl, Caleb Gattegno, Zoltan Dienes, Seymour Papert

Gave kids toys and gifts to learn in a motor way. 

Embodied Design: constructing means for constructing meaning. Make the concepts exist in the childrens minds. 
(CRUM giving them concepts so they may able to draw meaning, make into propositions, images, analogies etc) 
Experience first, analyze later: building mathematical concepts form perception, action, aesthetics, and common sense. 

Embodied Mathematical Reasoning: 
Terrence Tao: math professor at 24. 
Egg frying in a pan of oil
Jet ski riding  a wave in the ocean, 
Rolling on the floor, 

Enactivism (Varela, Thompson, and Rosch): "in a nutshell the enactive approach consists of two points:
1) Perception consists in perceptually guided action; and
2) cognitive structures emerge from the recurrent sensorimotor patterns that enable action to be perceptually guided."

example: proportion: wii sensors where the left side had a proportion of 1 and the right 2. So to make the screen green, right side had to be 2x the left. \\


\subsection{Discussion (4/13/2018)}
Mind-Body Problem: (What is the relationship between the mind and the body?
Monism: Mind and body are fundamentally the same
Dualism: Mind and body are distinct entities
Physical brain vs mind as in thoughts/memories/experiences/feelings

Emergent property: A property of a system which cannot be found in the system's individual constituents

Salt(NaCl) = Na (explosive not salty) + Cl (corrosive and not salty) 
salty

Is the consciousness emergent property? is this dualist or monist? 
Chinese room argument?

Mechanisms of consciousness according to Paul Thagard: Biological, electrical and chemical
consciousness isn't binary, many parameters such as drugs, can affect

Study consciousness: optical illusions, neuropathies (blindsight, visual neglect), brain activity (EEG), sleep/dreams

Role of visual system: Greyson's claim: "visual experience informs us about the world but it is not the sort of thing that can be accurate or inaccurate; it is a matter of construction"

Seeing stuff, or just abstracting away the important stuff and seeing what will help us survive. (construction home vs painting a picture) 


\section{Perspectives from Anthropology: GSI lecture: Tracy Brannstorm}
\textbf{Key points:}
\begin{itemize}
    \item overview of anthropology
    \item evolutionary psychology
    \item fieldwork: mind-body medicine in Northern Peru
    \item Contemporary scientific research
\end{itemize}

Sub-fields of anthropology
1) cognitive archaeology
studying the human mind from past societies throught their material remains
    groups of poeple living together tend to develop a shared view of the world and similar cognitive maps which in turn influence their material culture

2) biological
    Links social behavior to biology
    evolution of human brain and consciousness
    primate behavior 
Evolutionary psychology 
    evolutionary approach to understanding human behaviour and mind
    how and why did human intelligence and consciousness evolve? 
    evolution: the process of change in a species that occurs over many generation, through natural selection
    disease avoidance mechanism
    
3) Linguistic anthropology
    how do language and culture interact
    sapir-whorf hypothesis 
        "what surprises me most is to find that various grand generalizations of the western world, such as time, velocity, ad matter, are not essential to the construction of a consistent picture of the universe.."(whorf) 
    Daniel everrtt et al. (2008) 
        Piraha have no words for expressing exact quantities or colors
        suggests that language is a cultural invention rather than a linguistic universal

4) Cultural anthropology 
    looking for shared realities: knowledge, behavior, interactions, experience, language, and meaning
    
    "Reality is not simply out there, waiting to be uncovered. 'truth is not born nor is it to be found inside the head of an individual person, it is born between people collectively searching for truth in the process of their dialogic interaction,' Bakhtin wrote in 1929.. Nothing simply is itself, outside the matrix of relationships in which it appears. Instead, being is an act or event that must happen in the space between the self and world." - Abeba Birhane "Descartes wass wrong: 'a person is a person through other persons," (Aeon, April 2017) 
    
    \textbf{context:} how different social and cultural contexts shape a person in various ways = cognition, etc. 
    individuals respond to and interact with cultural materials and contexts, but culture is also responsive to , and created by, the ways that individuals enact it. The direction of influence is two-way-"looping"(hacking 1999) 
    Looking for the specific rather than the general
    (How something works for you depends on the context, the environment, you are in. Say a vitamin drink for example, it can help you depending on where you are, waht you are doing, and how you are doing, rather than just saying the vitamin drink doesn't work because it's phony doesn't mean it doesn't work, completely dependent on the person, shared reality and the visualizer. (Reason why placebos work and why somethings work for one while for others it doesn not) 
    
\textbf{What does the cultural antrhopologist Do?}

collect qualitative data (sometimes quantitative) 
    any information that can be captured that is not numerical in nature (i.e, how a person perceives a relationship) 
    
Fieldwork: interviews (participant) observation
    the researcher studies a society from the point of view of that society; the meaning that derive from their experiences
    researchers using their bodies and minds as primary tools
    "human experience cannot be measured in a lab. it's too private, fluid, specific, idiosyncratic, difficult to capture and predict, and a little wild." 

Margeret Mead and Gregory Bateson



\section{Cognitive Science and Digital Technologies: Guest Lecture Todd Davies, Stanford University Symbolic Systems Program}
Symbolic Systems analagous to Cognitive Science

Different environment of research: 20th century libraries books, 21th century internet
Anyone with internet can have access to research articles and materials. Much difficult before with few libraries with science research

Outline: 
technology and human well-being: quick overview
media psychology: case studies
deep dive = online diffusion and virality (Davies work) 
computational psychology

\subsection{technology and human well-being}
World GDP with technology has increased since more tech has been created especially since the 1700's up

Marshall McLuhan (1911 - 1980) 
technology drives history media as "extensions of (hu)man(ity)"
(has big influence on history, maybe drive history) 

"the medium is the message" 
changes scale of humanity activity 
example: the electric light as an information technology
(changed how we live where we could no traverse and do things in the night time wisth day time light source) 

Psychology of Influence 
McLuhan: global village (have influence not only through your neighborhood, but globally) 
Digital technology makes us more dependent on mental shortcuts - fatigue, rush, overload
gives power to exploiters, e.g. of social proof

Robert Cialdini's answer: resist the exploiters!
(when something is more popular, we tend to want to get it as well) 
(can be more persuaded to get items through tech, ex: advertisements) 

\subsection{Technology and well-being}
happiness has not increased in U.s. since 1946 
higher inequality, depression, and anxiety
Amish are happier than most people

Richard Easterlin (1974) 
money improves happiness dramatically for the poor 
for the non-poor it has little effect

Daniel Kahneman and Angus Deaton (2010) 
income gains improve happiness up to about \$75,000
beyond that, there is little improvement

Psychological mechanisms
loss aversion - we care much more about losing what we have than gaining something new
myopic decision making - pay attention to our change in position than our absolute position (examine how we are to the future, instead of overall) 
hedonic treadmill- adaptation to new wealth (whatever wealth we achieve, we adapt and any minor changes can make us unhappy) 
(running doesn't do anything but keep you from standing in place) (more wealth doesn't change the happiness you felt before) 

What is so special of "real" 
prevalence of robot caregivers
Sherry Turkle : digital technologies give us "moments of more, and lives of less" 

Clifford Nass and colleagues findings on scoial media 
High "media multitaskers" have more difficulty focusing than low - MMers
Result extends to young people more than adults 
young people who are heavy users of social media are poorer at reading faces, less confident about social interactions

bandwagon effect!!!!

Schematic diffusion patterns
broadcast vs viral
one node vs multi nodes
messenger important vs message important

Structural virality vs intrinsic virality (infectiosness) 

Main model in Goel et.al. (2016) assumes constant infectiousness (intrinsic appeal of the content/message) 
they say: "in other words, taking infectiousness as a a proxy for quality, in our simulations the largest and most ...



\section{Discussion 4/20/2018}
Last week: defined a metric of consciousness utilizing awareness (are we self aware?) 
Thomas Nagel: a mental state is conscious just if there is "something it's like" to have that mental state
(because i can imagine what it is for an apple to be red, i have a conscious phenomenon of viewing something as red.) 
if i have something to describe what "something is like", then i have a sort of consciousness
(unconscious mental state: like under general anesthetic for say wisdom teeth surgery. Can't recall the experience)

What features can we ascribe to consciousness? (assuming definition above) 

reductionism: can reduce a system to it's constituents, it's parts, i can figure something out about the system from those parts. 
("what is it like to be a bat? (MCB reading)) 

\subsection{Hard vs easy problem}

Easy Problem : specifying mechanisms that explain how functions are performed
    How do  we see the color red -> red light enters the eye; cones in eye respond; retinal cells send info to the rest of the visual sytem
    
Hard Problem: explaining how and why we have phenomenal experiences
    Why does red look red?
    why are we conscious of sensory information at all?
    
Reporting consciousness
"Methodological Problem": how do we distinguish neural mechanisms relelvant to our access of consciousness from those directly relevant to consciousness itself?

Phenomenal Consciousness vs Cogntive Accessibility

Rich View: the conscious visual experience contains a richer array of information than is accessed
Sparse View: the information we can access is the only information that we are consciously aware of
Change blindness: the thing changing in the picture from one to the other

Sperling experiment 

Tries to show rich view

\subsection{Ebodied Learning}
Embodied cognition: George Lakoff's idea that you need the sensory motor experience of something (pass the cup) 
Enactivism: cognition arises through interaction between an organism and its environment
1) perception consists in perceptually guided action
2) cognition structures emerge from recurrent sensorimotor patterns that enable actions to be perceptually guided

green screen with proportions


\section{Lecture 4/24/2018}
"you know far less about yourself than you feel you do"

Regrow body parts, mind reading, neuroprosthetics, telekinesis, mind control 

\subsection{STEM CELLS}
stem cells can turn into any other cells
blastocysts (5 day old embryo): where we extract stem cells
\textbf{Properties of STEM Cells}
\begin{itemize}
    \item self renewal: replicate itself indefinitely
    \item differentiation: turn any stem cell into any type of cell in the human body
\end{itemize}

\subsection{Induced Pluripotent Stem (iPS) cells:}
\textbf{Geneticallly engineering new stem cells } 
steps: biopsy, reprogramming factors, induction of pluripotency, iPS celss, differentiation, (transplantation (patients) or disease affected cell type (model))
CRISPR/Cas9 genome editing tool

\subsection{Mind Reading}
Jack gallang using fMRI to recreate images in brain

\subsection{Neuroprosthetics}
silicon chips to implant memories? augment lost functions!
biocopatibility: will our brains take in a foreighn device long term?
Ted Burger: made mathematical theorem describing how neurons move through hippocampus and how we convert STM to LTM. 

\subsection{Telekinesis}
BMI (brain mind interfaces) 
cochlear implants, control arm with mind 

\subsection{Mind Control}
using light technology to move a different organism 
Optogenetics: turn on lights to activate geneticallky modified neurons in the brain

Rejuventation vs regeneration
parabiosis
connects mouses together and young mouse becomes old and old mouse becomes young
GDF 11 in blood helps make that
fountain of youth in blood

Combat against age related diseases to extend our lives

\section{Last Lecture}
Course overview and final thoughts \\ 

2 main approaches,CRUM model paul thagard vs cog sci 1
Can bring a moral compass to the computer design process

\section{Discussion}

Read chapter 9 
Four major subfields: focus on biological and cultural anthropology
evolutionary psychology: how did our minds evolve t oprocess information and carry out tasks in the way that they do? 
good dads and bad cads
very good parent protecting one child
or making as many as possible so that one may  survive and pass on genes
ethology (study of beahviour) 

Cultural anthropology: truth is not born nor is it to be found inside the head of an individual person, it is born between people cllectively searching for truth" 

cultural anthropologists study shared realities and how paeople make meaning within various contexts
Methodology: 1) participation observation 2) interviews 3) surveys 
collected data qualitative
WEIRD (western, educated participants from industrialized, rich, democratic countries) 

Digital technolog in social sciences 
the medium in which you send informmation has an impact on how the message is received
the scale of human activity
online diffusion
avg depth of nodes index
$v(t) = \frac{1}{n(n-1)} \Sigma_{i = 1}^n \Sigma_{j = 1}^n d_{ij}$

(if someone finds people that are like minded they will all think the same even tho it's not necessarily true. 
Broadcast vs Viral
Broadcast: one node to everyone else
Viral: message is high quality that everyone will get it no matter where it came from. 
Doesn't explain how it works, just shows what happens kind of

used petition success as an example of the virality vs broadcast
exceed ratio: "broadcast"iness
Glabal peak only exceed ratio: same thing but for global peak

FDSD: intrinsic virality measurement

successful petitions had lower broadcastiness and higher intrinsic virality

CRUM Model 
"The central hypothesis of cognitive science is sthat thinking ca best be understood in ters of representational structures in the mind and computational procedures that operate on those structures" 
When you think about something, you represent the ideas in some way and you operate on these ideas: 

How do you represent these ideas? (CRAPI) 
concept, rules, analogies, propsition, image


I can represent a dog as a concept. Concept of dog has several properties such as: breed, name, color, x-position, y-position
to think about the dog walking across the street, i will operate on my instantiation oof the dog object. 

neuroscience: 
structure of biology, nerves (dendrietes, nucleus soma,mylelin, axons
synaptic cleft, ion channels, Na+/ k+/ ca++/ and cl- 
action potentinoals: IPSPs, EPSPs, refactory period, thresholds, resting state
organization of biology: four lobes, temporal, frontal, parietal, occiptal, thalamus, hippocampus
optic chiasm, contralateral signals of body and visual fields. 
somatosensory mappings (wilder penfiled) 
rat brain lesions 

neural imaging techniques, good bad, spatial temporal

visual systems: retina, lgn(thalamus) , v1, v2, v3, v4 ventral (what) , v5 dorsal (where), 
P cells, mcells, nonM-nonP cells, 
Parvocellular, magnocellular, koniocellular

hubel and wiesel: simple cells(edge detection and orientation) vs coplex cleels (direction, motion) :

Psycholoy 
roger sperry experiment (split-brain experiment) 

memory systems: multi modal model and tulving model 

chunking to remember easier
only remember 7 +- 2 things at a time

methods of catgorization:
exemplar approach:compare new object to all instance of a category in memory
feature approach: feature approach: determine necessary and sufficient features category
prototype approach: compare new object to average instace of a category

category levels: superordinate, basic-level, subordinate
Radical behavioiurism: treat mind as  ablack box; (BF skinner) 
classical conditioning, operant condition (radical) 

Linguistics
 Poverty of the stimulus: children do not have enough data to learn natural language so they should be hard wired to learn grammtical rules (noam chomsky) recursion, grammatically correct, semantically meaningless
 
 sapir-whorf hypothesis: language shapes thought (linguistic relativism) 
teenie matlock (fictive motion and aspect) 

wernickes and brocas area
Kim et al. study showing early and late bilinguals (difference of locations in wernickes and brocas for late and early bilinguals) 

emobided cognition: cognition is constructed and shaped by sensorimotor functions
George lakoff: emobided cognition in metaphor
High-level abstract idea use metaphors 

computer science 
AI: make robots or machines think and act rationally like humans

strong AI vs weak AI hypotheses
chinese room argument: four replies
systems reply: maybe the man in the room doesn't know chinese, but the room as a whole system does: emergent
robotreply: what if the man in the room was actually in a robo so it has a body
brain-simulator reply: what if we wrote a program that exactly mimicked the biological function of a brain in a native chinese speaker
other minds reply: the only way you can tell that other people understand chineses is 

Neural networks: computing systems in spired by biological neural networks. consist of neurons, connection /weights
mcullock pitts neuron: 
perceptron is an MP neuron (linear, binary) 

machine learning: gett computters to change their behavior without explicitly programning the
deep learning: has several layers and tries to optimize how to repesent featres autonomously
(how clean is your data??) 





\end{document}